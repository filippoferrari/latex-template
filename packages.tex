% Fixes some weird errors
\usepackage{etex}
% Tell me about my LaTeX bad practices
\usepackage[l2tabu, orthodox]{nag}
% For line brakes in tables
\usepackage{tabularx}
% For double lines in tables
\usepackage{hhline}
% For multirow
\usepackage{booktabs}
\usepackage{multirow}
% For the split environment
\usepackage{amsmath}
% For tabs in verbatim
\usepackage{moreverb}
% For algorithms
\usepackage{algorithm}
\usepackage[noend]{algorithmic}
\renewcommand{\algorithmiccomment}[1]{\hfill// #1}
\renewcommand{\algorithmicrequire}{\textbf{Input:}}
\renewcommand{\algorithmicensure}{\textbf{Output:}}
% For splitting long lists into columns
\usepackage{multicol}
% For no multicol on the kindle
\usepackage{environ}
\newif\ifmulticols
\NewEnviron{mymulticols}{
\ifmulticols
  \begin{multicols}{2}
    \BODY
  \end{multicols}
\else
  \BODY
\fi}
% Means you don't have to put \\ to start a new line.
\usepackage[parfill]{parskip}
% Make LaTeX pretty with better kerning etc
\usepackage{microtype}
% tikz
\usepackage{tikz}
% For graphs
\usepackage{pgfplots}
% For placing figures *exactly* where I want them
\usepackage{float}
% For standalone tikz
\usepackage{standalone}
% For superscript
\usepackage[super]{nth}
% For page break in align tags
\allowdisplaybreaks
% For syntactic trees
\usepackage{qtree}
% For fancy parenthesis
\usepackage{stmaryrd}
% For proofs
\usepackage{bussproofs}
% For fancy math letters
\usepackage{ mathrsfs }
\usepackage{ amssymb }
% For monospace fonts
\newcommand{\mo}[1] {\texttt{#1}}

% Set up listings
\usepackage{color}
\usepackage{xcolor}
\usepackage{listings}
\usepackage{caption}
\usepackage[T1]{fontenc}
\usepackage[variablett]{lmodern}
%\input{scala_listing.tex}
%\definecolor{OliveGreen}{RGB}{60,128,49}
%\DeclareCaptionFont{white}{\color{white}}
%\DeclareCaptionFormat{listing}{\colorbox{gray}{\parbox{\textwidth}{#1#2#3}}}
%\captionsetup[lstlisting]{format=listing,labelfont=white,textfont=white}
%\lstset{language=Scala,captionpos=b,tabsize=1,frame=lines,keywordstyle=\color{blue},commentstyle=\color{OliveGreen},stringstyle=\color{red},numbers=left,numberstyle=\tiny,numbersep=1pt,breaklines=true,showstringspaces=false,basicstyle=\footnotesize,emph={label}}

\DeclareMathOperator{\dom}{dom}